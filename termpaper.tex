% This file is based on the "sig-alternate.tex" V1.9 April 2009
% This file should be compiled with V2.4 of "sig-alternate.cls" April 2009

\documentclass{sig-alternate}

\usepackage{url}
\usepackage{color}
\usepackage{enumerate}
\usepackage{balance}
% Define BibTeX command
\def\BibTeX{{\rm B\kern-.05em{\sc i\kern-.025em b}\kern-.08em
    T\kern-.1667em\lower.7ex\hbox{E}\kern-.125emX}}
\permission{}
\CopyrightYear{2017}
\crdata{0-00000-00-0/00/00}
\begin{document}

\title{Data Analytics and Cognitive Computing Group Project}
\subtitle{Detecting Patterns in ...????}
\numberofauthors{2}
\author{
\alignauthor
Victoria Sardelli
\alignauthor
Orens Xhagolli
}
\date{22 September 2017}
\maketitle
\begin{abstract}
  
\end{abstract}

\section{Introduction}
\label{introduction}

Use three or four paragraphs to present an introduction to your termpaper.

Alsp, provide a roadmap for the remaining sections of the paper. For
example, you can state that Section \ref{motivation} presents a
motivation for the term paper, and section \ref{related work} presents your
literature survey of the work done in your target area. Section
\ref{current approaches} discusses different approaches that have been
taken to address cognitive computing, and \ref{comparative analysis}
presents a comparative analysis of the approaches. Section
\ref{lessons learned} should discuss---in sufficient depth---what you
learned from this term paper.  Section \ref{current status and future
  work} describe the current state of the project and what else
could be done in the future.

{\bf Note 1: This specific file is a generic template so the
  section titles may not fit your specific needs so feel
  free to change them as needed.}

{\bf Note 2: Please change the words so that you use your own word,
  not these suggested words. This is important because the paper needs
  to be in your own words!}

\section{Motivation}
\label{motivation}

Use this section to motivate why your term paper. Describe the
fundamental issues being explored in the term paper, and what kinds of
comparisons

\section{Related Work}
\label{related work}

Use this section to describe what other people have been doing to
in this space. Make sure your literature survey is fairly
complete, and you must cite your sources correctly per ACM style
guidelines (and of course, you need to use \LaTeX and \BibTeX correctly).

\section{Current Approaches}
\label{current approaches}

Discuss each of the approaches/products/projects that you are comparing in this paper.

\subsection{IBM Watson}
\label{IBM Watson}
Provide the basic features, strengths and weaknesses of this approach. 


\subsection{Approach 2}
\label{approach 2}
Provide the basic features, strengths and weaknesses of this approach. 


\subsection{Approach 3}
\label{approach 3}
Provide the basic features, strengths and weaknesses of this approach. 



\subsection{Approach 4, if any}
\label{approach 4}
Provide the basic features, strengths and weaknesses of this approach. 



\section{Comparative Analysis}
\label{comparative analysis}

This section will present a comparison of the different approaches to
cognitive computing, which you discussed in Section \ref{current approaches}.


\section{Lessons Learned}
\label{lessons learned}

Describe what you learned from the term paper;
this section, like the others, plays a critical component in
determining your final grade.

\section{Current Status \& Future Work}
\label{current status and future work}

Use this section to describe the current status of your work
and what else needs to be done. Also, discuss what further directions
your work can take by others.



The next subsection is meant to provide you with some help
in dealing with figures, tables and citations, as these are hard for
folks new to \LaTeX. 

{\bf And please delete the following subsection before you make any submissions!}

\subsection{Tables, Figures, and Citations/References}

Tables, figures, and citations/references in technical
documents need to be presented correctly. As many students
are not familiar with using these objects, here is a quick
guide extracted from the ACM style guide.

\begin{table}
\centering
\caption{Feelings about Issues}
\begin{tabular}{|l|r|l|} \hline
Flavor&Percentage&Comments\\ \hline
Issue 1 &  10\% & Loved it a lot\\ \hline
Issue 2 &  20\% & Disliked it immensely\\ \hline
Issue 3 &  30\% & Didn't care one bit\\ \hline
Issue 4 &  40\% & Duh?\\ \hline
\end{tabular}
\end{table}


First, note that figures in the term paper must be original,
that is, created by the student: please do not cut-and-paste
figures from any other paper you have read. Second, if you
do need to include figures, they should be handled as
demonstrated here. State that Figure \ref{sample graphic} is
a simple illustration used in the ACM Style sample
document. Figures are never below or above the
text. Incidentally, in proper technical writing (for reasons
beyond the scope of this discussion), table captions are
above the table and figure captions are below the figure.

\begin{figure}[htb]
\label{sample graphic}
\begin{center}
\includegraphics[width=1.5in]{fly.jpg}
\caption{A sample black \& white graphic (JPG).}
\end{center}
\end{figure}

Finally, citing documents needs to be done properly too. For
example, a paper by Mic Bowman, Saumya K. Debray, and Larry
L. Peterson could be cited as Bowman, Debray, and Peterson
\cite{bowman:reasoning}. A set of papers could collectively
be cited as the literature in this area consists of several
interesting papers
\cite{braams:babel,clark:pct,herlihy:methodology}.

You will find the BibTeX entries needed for many papers being cited,
otherwise you can write your own versions easily and add them to the
$report.bib$ file in the folder. There are many sample bibtex
template files that can be used to model your own references.

The list of all references will be generated in ACMRef
standard style using the \LaTeX{}/\BibTeX{}. Note that you
need to first the following sequence to get the paper
compiled correctly:

\begin{enumerate}
\item {\tt latex} {\em termpaper}
\item {\tt bibtex} {\em termpaper}
\item {\tt latex} {\em termpaper}
\item {\tt latex} {\em termpaper}
\end{enumerate}

\bibliographystyle{abbrv}
\bibliography{termpaper}
% You must have a proper ".bib" file
%  and remember to run:
% latex bibtex latex latex
% to resolve all references
\balance
\end{document}








